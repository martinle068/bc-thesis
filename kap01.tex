%%% Fiktivní kapitola s ukázkami sazby

\chapter{Nápověda k~sazbě}

\section{Úprava práce}

Vlastní text práce je uspořádaný hierarchicky do kapitol a podkapitol,
každá kapitola začíná na nové straně. Text je zarovnán do bloku. Nový odstavec
se obvykle odděluje malou vertikální mezerou a odsazením prvního řádku. Grafická
úprava má být v~celém textu jednotná.

Práce se tiskne na bílý papír formátu A4. Okraje musí ponechat dost místa na vazbu:
doporučen je horní, dolní a pravý okraj $25\,\rm mm$, levý okraj $40\,\rm mm$.
Číslují se všechny strany kromě obálky a informačních stran na začátku práce;
první číslovaná strana bývá obvykle ta s~obsahem.

Písmo se doporučuje dvanáctibodové ($12\,\rm pt$) se standardní vzdáleností mezi řádky
(pokud píšete ve Wordu nebo podobném programu, odpovídá tomu řádkování $1,5$; v~\TeX{}u
není potřeba nic přepínat).

Primárně je doporučován jednostranný tisk (příliš tenkou práci lze obtížně svázat).
Delší práce je lepší tisknout oboustranně a přizpůsobit tomu velikosti okrajů:
$40\,\rm mm$ má vždy \emph{vnitřní} okraj. Rub titulního listu zůstává nepotištěný.

Zkratky použité v textu musí být vysvětleny vždy u prvního výskytu zkratky (v~závorce nebo
v poznámce pod čarou, jde-li o složitější vysvětlení pojmu či zkratky). Pokud je zkratek
více, připojuje se seznam použitých zkratek, včetně jejich vysvětlení a/nebo odkazů
na definici.

Delší převzatý text jiného autora je nutné vymezit uvozovkami nebo jinak vyznačit a řádně
citovat.

\section{Jednoduché příklady}

K~různým účelům se hodí různé typy písma.
Pro běžný text používáme vzpřímené patkové písmo.
Chceme-li nějaký pojem zvýraznit (třeba v~okamžiku definice), používáme obvykle
\textit{kurzívu} nebo \textbf{tučné písmo.}
Text matematických vět se obvykle tiskne pro zdůraznění \textsl{skloněným (slanted)} písmem;
není-li k~dispozici, může být zastoupeno \textit{kurzívou.}
Text, který je chápan doslova (například ukázky programů) píšeme \texttt{psacím strojem}.
Důležité je být ve volbě písma konzistentní napříč celou prací.

Čísla v~českém textu obvykle sázíme v~matematickém režimu s~desetinnou čárkou:
%%% Bez \usepackage{icomma}:
% $\pi \doteq 3{,}141\,592\,653\,589$.
%%% S \usepackage{icomma}:
$\pi \doteq 3,141\,592\,653\,589$.
V~matematických textech je často lepší používat desetinnou tečku
(pro lepší odlišení od čárky v~roli oddělovače).
Nestřídejte však obojí.
Numerické výsledky se uvádějí s~přiměřeným počtem desetinných míst.

Mezi číslo a jednotku patří úzká mezera: šířka stránky A4 činí $210\,\rm mm$, což si
pamatuje pouze $5\,\%$ autorů. Pokud ale údaj slouží jako přívlastek, mezeru vynecháváme:
$25\rm mm$ okraj, $95\%$ interval spolehlivosti.

Rozlišujeme různé druhy vodorovných čárek:
červeno-černý (tzv. spojovník),
strana 16--22 (střední pomlčka),
$45-44$ (matematické minus),
a~toto je --- jak se asi dalo čekat --- vložená věta ohraničená dlouhými pomlčkami.

V~českém textu se používají \uv{české} uvozovky, nikoliv ``anglické''.

% V tomto odstavci se vlnka zviditelňuje
{
\def~{{\tt\char126}}
Na některých místech je potřeba zabránit lámání řádku (v~\TeX{}u značíme vlnovkou):
u~předložek (neslabičnych, nebo obecně jednopísmenných), vrchol~$v$, před $k$~kroky,
a~proto, \dots{} obecně kdekoliv, kde by při rozlomení čtenář \uv{ško\-brt\-nul}.
}