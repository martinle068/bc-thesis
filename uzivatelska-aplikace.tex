\chapter{Uživatelská aplikace}

\lstset{
    frame=single,
    breaklines=true,
    basicstyle=\ttfamily\small
}

Abychom ilustrovali, že Kodemorph funguje správně, je v rámci této práce, pro testovací účely, poskytnuta demo uživatelská aplikace. Aplikace využívá Svelte, FastAPI a MongoDB a lze sloužit jako referenční řešení. Aby byl uživatel schopen vytvořit vlastní aplikaci, která je kompatibilní s nástrojem Kodemorph, musí znát omezení programu a dodržovat následující zásady.

\section{Předpoklady a požadavky prostředí}

\begin{itemize}
    \item \textbf{Docker Desktop} -- musí být nainstalován a spuštěný
    \item \textbf{Git} -- pro klonování repozitáře a zobrazení změn po aplikování Kodemorph transformací
    \item \textbf{Node.js} (verze 24+)
    \item \textbf{Libovolný jazyk pro backend} 
    \item \textbf{MongoDB} (pro MongoDB režim)
\end{itemize}

\section{Konfigurace}

\subsection{Konfigurační soubor (kodemorph.config.json)}

Soubor musí existovat v kořenové složce projektu a uživatel musí specifikovat cestu obsahující zdrojový kód aplikace. Kodemorph automaticky najde všechny \texttt{.svelte} soubory v zadaném adresáři a jeho podadresářích.

\begin{lstlisting}[caption=Konfigurace Kodemorph]
{
    "svelteDirectory": "src/components"
}
\end{lstlisting}

\subsection{Proměnné prostředí (.env soubor)}

Pro MongoDB režim je třeba vytvořit \texttt{.env} soubor s informacemi pro připojení k databázi:

\begin{lstlisting}[language={}, caption=Proměnné prostředí]
MONGODB_URI=mongodb://localhost:27017
MONGODB_DATABASE=your_database
MONGODB_COLLECTION=your_collection
\end{lstlisting}

\section{Požadavky na Svelte soubory}

\subsection{Povinná anotace datových zdrojů}

\textbf{DŮLEŽITÉ:} Každý Svelte soubor, který má být transformován, MUSÍ obsahovat anotaci \texttt{@kodemorph} u proměnných s daty. Anotace se vkládá jako komentář nad proměnnou.

\textbf{Správně:}
\begin{lstlisting}[language=HTML, caption=Správné použití anotace, escapeinside={(*@}{@*)}]
<script>
    (*@\texttt{// @kodemorph}@*)
    export let users = [];
</script>

<div>
    {#each users as user}
        <p>{user.username}</p>
    {/each}
</div>
\end{lstlisting}

\textbf{Špatně (nebude fungovat):}
\begin{lstlisting}[language=HTML, caption=Chybějící anotace, escapeinside={(*@}{@*)}]
<script>
    export let users = [];  (*@\texttt{// Chybí @kodemorph}@*)
</script>
\end{lstlisting}

\subsection{Vzor kódu, který Kodemorph očekává}

\begin{lstlisting}[language=HTML, caption=Typický vzor Svelte komponenty, escapeinside={(*@}{@*)}]
<script>
    (*@\texttt{// @kodemorph}@*)
    export let users = [];
</script>

<div>
    {#each users as user}
        <div>
            <p>Name: {user.username}</p>
            <p>Email: {user.email}</p>
            <p>Role: {user.role}</p>
            {#each user.tags as tag}
                <span>{tag}</span>
            {/each}
        </div>
    {/each}
</div>
\end{lstlisting}

\pagebreak
\section{Datová struktura}

\subsection{MongoDB kolekce}
Struktura dokumentů v MongoDB musí být konzistentní napříč všemi dokumenty:

\begin{lstlisting}[caption=Příklad MongoDB dokumentu]
{
    "_id": "507f1f77bcf86cd799439011",
    "username": "john_doe",
    "email": "john@example.com",
    "role": "admin",
    "tags": ["active", "premium"]
}
\end{lstlisting}

\subsection{Omezení struktury}
\begin{itemize}
    \item Nekonsistentní struktura (některé dokumenty mají pole, jiné ne) -- nepodporováno
    \item Smíšené datové typy v poli (\texttt{["string", 123, \{obj: true\}]}) -- nepodporováno
    \item Globální identifikátor dokumentu \texttt{\_id} na kořenové úrovni -- požadováno
\end{itemize}

\section{Požadavky na Backend}

Kodemorph podporuje jakýkoliv backend (FastAPI, Flask, Express, Spring Boot, atd.), pokud splňuje následující požadavky:

\subsection{Práce s raw JSON/BSON}
Backend musí pracovat přímo s \textbf{raw JSON} nebo \textbf{BSON} strukturami bez vlastních modelů nebo ORM transformací.

\textbf{Správně (raw JSON/BSON):}
\begin{lstlisting}[language=Python, caption=Správná implementace s raw JSON, escapeinside={(*@}{@*)}]
@app.get("/api/users")
async def get_users() -> List[Dict[str, Any]]:
    users = []
    cursor = users_collection.find()  (*@\texttt{\# Raw MongoDB query}@*)
    async for document in cursor:
        users.append(document)  (*@\texttt{\# Vrací raw document}@*)
    return users

@app.post("/api/users")
async def create_user(user_data: Dict[str, Any]):
    result = await users_collection.insert_one(user_data)
    return serialize_doc(result)
\end{lstlisting}

\textbf{Špatně (ORM modely):}
\begin{lstlisting}[language=Python, caption=Problematická implementace s ORM modelem, escapeinside={(*@}{@*)}]
(*@\texttt{\# Toto NEBUDE fungovat - použití Pydantic modelu}@*)
class UserModel(BaseModel):
    username: str
    email: str
    role: str

@app.post("/api/users")
(*@\texttt{\# Model definuje strukturu}@*)
async def create_user(user: UserModel):  
    (*@\texttt{\# Kodemorph nemůže změnit strukturu, je fixní v modelu}@*)
    ...
\end{lstlisting}

Toto omezení je nutné z důvodu, že již prosté přejmenování jednoho klíče struktury způsobí nekompatibilitu s ORM modelem. Zbylé transformace by způsobily mnohem závažnejší problémy, které by vyžadovaly dynamickou transformaci backend části aplikace a to Kodemorph neumožňuje. 

\subsection{Žádná transformace při čtení/zápisu}
Backend \textbf{musí předávat JSON přesně tak, jak je uložen v databázi}. Žádné mapování, konverze ani úpravy struktury při čtení nebo zápisu dat.

\textbf{Zakázáno:}
\begin{itemize}
    \item Přejmenování polí při serializaci/deserializaci
    \item Skrývání nebo filtrování polí
    \item Automatické konverze typů (např. datum na string)
\end{itemize}
Je-li jakákoliv transformace nutná, uživatel musí na základě změny zvážit manuální zásah do struktury dokumentů, tzn. upravit všechny dokumenty v databázi, aby byly konzistentní.

\section{Známá omezení}

\subsection{Transformace ``wrap as list''}
\begin{itemize}
    \item MongoDB transformace funguje
    \item Svelte transformace není implementována
    \item Po této transformaci je nutná ruční úprava Svelte kódu. Obvykle je nutná enumerace nově vzniklého seznamu a to vyžaduje zásah uživatele.
\end{itemize}

\section{Vytvoření demo aplikace}

\subsection{Struktura}

\begin{verbatim}
demo-app/
  docker-compose.yml
  init-mongo.js
  backend/
    main.py (FastAPI)
    requirements.txt
  frontend/
    src/
      components/ (Svelte s @kodemorph)
      pages/
\end{verbatim}
